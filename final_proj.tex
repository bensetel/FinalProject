\documentclass{article}
\usepackage{hyperref}
\usepackage{fancyhdr}
\usepackage{amssymb}
\usepackage{amsmath}


\title{Linear Algebra in Octave}
\author{Benjamin J. Setel}
\date{\today}

\lhead{Ben Setel}
\chead{Linear Algebra in Octave}
\rhead{CS 177}

\begin{document}
\pagestyle{fancy}

\maketitle{}
\tableofcontents{}

\section{Background}
For this tutorial, we will be using Octave's official linear algebra package \footnote{Available online at: http://octave.sourceforge.net/linear-algebra/index.html}. The techniques of linear algebra discussed here can be powerful tools for solving linear systems equations of the form:
\[a_1x_1+a_2x_2+\ldots+a_nx_n=c\] where $a_1 \ldots a_n$ are constants, and $x_1 \ldots x_n$ are variables in $\mathbb{R}$ (that is, they are real numbers).  Many of the operations performed in linear algebra, such as finding the determinant of a matrix, matrix multiplication, and diagonalizaiton can be performed quickly and accurately by computers.  In this tutorial, we will explore the many functions that octave provides to solve problems in linear algbra.
\section{A (very brief) Crash Course In Linear Algebra}
As previously mentioned, linear algebra concerns solving systems of linear equations.   One can put the coeffecients of such ssystems into a matrix.  For example, the system:
\begin{align*}
a_1x_1+a_2x_2 &= n_1 \\
b_1x_1+b_2x_2 &= n_2
\end{align*}
Produces the $2x2$ matrix:
\[
\left[
\begin{array}{cc|c}
a_1 & a_2 & n_1 \\
b_1 & b_2 & n_2 
 \end{array}
 \right]
\]
Note that each row corresponds to an equation and that the line seperates our coefficients from our solutions.  If we assume that these solutions are all 0, we can omit that column.
\section{Determinant}
A precise definition of the determinant requires a course in linear algebra (for Brandeis students, Math $22a$ is an excellent course$\ldots$).  Informally, it can be thought of as multiplying certain entries of a matrix together, and summing the results of these multiplications together in a specific way.  For a $3 \times 3$ matrix
\[A =
\left[
\begin{array}{ccc}
a_{11} & a_{12} & a_{13} \\
a_{21} & a_{22} & a_{23} \\ 
a_{31} & a_{32} & a_{33} 
 \end{array}
 \right]
\]
the determinant of $A$ is:
\[ det A = a_{11} \cdot a_{22} \cdot a_{33} + a_{12} \cdot a_{23} \cdot a_{31} + a_{13} \cdot a_{21} \cdot a_{32}  - a_{11} \cdot a_{23} \cdot a_{32} - a_{12} \cdot a_{23} \cdot a_{32} - a_{13} \cdot a_{22} \cdot a_{31}\]
As $A$ gets larger, the number of such terms grows as the factorial of the size of $A$.
Note that determinant is only defined on a ``square'' matrix (i.e. a matrix with that has an equal number of rows and columns).
\subsection{Finding the Determinant with Octave}
We now proceed to find such as determinant using Octave's linear algebra package.  The command:
\begin{verbatim}
A = [1 2 1; 4 3 6; 3 2 1]
\end{verbatim}
creates the $3x3$ matrix:
\[
\left[
\begin{array}{ccc}
1 & 2 & 1 \\
4 & 3 & 6 \\
3 & 2 & 1
 \end{array}
 \right]
\]
To find it's determinant, we enter the command \verb|det(A)|.  In this case we see that \verb|det(A)|$ = 18$.  Using the formula given above, you can see that this is indeed the determinant of the matrix. 
\subsection{Exercise One}
Using octave's \verb|det| function, find the determinant of the following matricies:
\begin{enumerate}
\item \[
        \left[
          \begin{array}{ccc}
            1 & 0 & 0 \\
            0 & 1 & 0 \\
            0 & 0 & 1
          \end{array}
        \right]
      \]
(Note that such a matrix is called the \emph{identity} matrix because of its multiplicitave properties).
\item
\[
        \left[
          \begin{array}{ccc}
            2 & 0 & 0 \\
            0 & 2 & 0 \\
            0 & 0 & 2
          \end{array}
        \right]
      \]
\item
\[
        \left[
          \begin{array}{ccc}
            1 & 0 & 1 \\
            1 & 1 & 0 \\
            1 & 0 & 1
          \end{array}
        \right]
      \]
\end{enumerate}
\section{Matrix Multiplication}
Multiplying two matricies is not as intuitive as multiplying two numbers.  For those without the background, the wikipedia page about it explains it quite well. Matrix multiplication in octave uses the * operator.  Sticking with our example
\[A=
\left[
\begin{array}{ccc}
1 & 2 & 1 \\
4 & 3 & 6 \\
3 & 2 & 1
 \end{array}
 \right]
\]
let's compute AA.  We enter \verb|A*A|, and get the matrix:
\[
\left[
\begin{array}{ccc}
12 & 10 & 14\\
34 & 29 & 28 \\
14 & 14 & 16
 \end{array}
 \right]
\]
which is indeed the correct answer.
\\If we set
\[B=
\left[
\begin{array}{ccc}
1 & 2 & 4\\
4 & 9 & 8 \\
5 & 7 & 6
 \end{array}
 \right]
\]
then we can compute $AB$ with the command \verb|A*B|  which gives us:
\[
\left[
\begin{array}{ccc}
14 & 27 & 26\\
46 & 77 & 76 \\
16 & 31 & 34
 \end{array}
 \right]
\]

\subsection{Dot Product}
We can also take the dot product of rows or columns of matricies (or more generally, any two vectors). If we want to dot the first rows of A and B, we can do so with the command: \verb|dot(A(1,1:3), B(1,1:3))|, which equals $9$.

\subsection{Exercise Two}
Multiply the following matricies, then take the dot products of their second rows.
\begin{enumerate}
\item
  \[
  \left[
  \begin{array}{ccc}
    1 & 13 & 40\\
    5 & -3 & 8 \\
    -9 & 2 & 7
  \end{array}
  \right]
  \left[
  \begin{array}{ccc}
    1 & 7 & 29\\
    4 & 6 & 10 \\
    6 & -3 & 0
  \end{array}
  \right]
  \]
\item
  \[
  \left[
  \begin{array}{ccc}
    1 & 0 & 0\\
    0 & 1 & 0 \\
    0 & 0 & 1
  \end{array}
  \right]
  \left[
  \begin{array}{ccc}
    1 & 2 & 3\\
    4 & 5 & 6 \\
    7 & 8 & 9
  \end{array}
  \right]
  \]
\end{enumerate}
\section{Eigenvalues}
Finding the eigenvalues of a matrix is straightforward in octave.  The command is \verb|eig(A)| where A is a square matrix.
\\E.g. For
\[C=
\left[
\begin{array}{ccc}
1 & 0 & 1\\
1 & 2 & 0 \\
1 & 0 & 1
 \end{array}
 \right]
\]
\verb|eig(C)|$= [2,2,0]$
\\(Note that in this case $C$ has characteristic polynomial: $-x^3+4x^2-4x$)

\section{Inverse, Solving Systems of Equations}
If we have a system of equations of the form:
\begin{align*}
a_{11}x_1+a_{12}x_2 + a_{13}x_3 &= b_1 \\
a_{21}x_1+a_{22}x_2 + a_{23}x_3 &= b_2 \\
a_{31}x_1+a_{32}x_2 + a_{33}x_3 &= b_3 \\
\end{align*}
then we can write this as $AX = B$.  With octave, we can solve such a system in two ways.
\\First, if $A$ is invertible, then we have $X=A^{-1}B$.  We can calculate the inverse with the command \verb|inv(A)|.  However, we can solve such a system even if $A$ is \emph{not} invertible.  In this case we use the ``left division'' operator: $\backslash$  In this case we have \verb|A\B = X|  
\\Octave is able to compute this solution \emph{without} inverting $A$.\\
If we have
\[A =
\left[
\begin{array}{ccc}
0 & 1 & 1\\
1 & 1 & 1 \\
1 & 1 & 1
 \end{array}
 \right]
\]
and
\[B =
\left[
\begin{array}{c}
3\\
6\\ 
9\\ 
 \end{array}
 \right]
\]
then 
\[A\backslash B'= X =
\left[
\begin{array}{c}
4.5\\ 
1.5\\ 
1.5\\ 
 \end{array}
 \right]
\]
Note that $B'$ is the transpose of $B$, which we use instead of $B$ so that the dimensions of $A$ and $B$ are compatible. 
\\In this case even though $A$ is not invertible, we could still find the answer.
\subsection{Exercise Three}
Solve each of the following systems of equations for $x_1, \ldots ,x_n$
\begin{enumerate}
\item
  \begin{align*}
    3x_1 + 2x_2 + x_3 &= 3 \\
           7x_2 + 2x_3 &= 9 \\
    x_1 + x_3 &= 10 \\
  \end{align*}
\item
  \begin{align*}
           x_2 + 2x_3 &= 6 \\
    5x_1 - 2x_2 + 2x_3 &= 8 \\
    x_1 + x_2 &= 16 \\
  \end{align*}
\end{enumerate}
\newpage{}
\section{Solutions}
\subsection{Exercise One}
\begin{enumerate}
  \item 
    \begin{verbatim}
      > A = [1 0 0; 0 1 0; 0 0 1];
      > det(A)
      ans = 1
    \end{verbatim}
  \item
    \begin{verbatim}
      > A = [2 0 0; 0 2 0; 0 0 2];
      > det(A)
      ans = 8
    \end{verbatim}
  \item
    \begin{verbatim}
      > A = [1 0 1; 1 1 0; 1 0 1];
      > det(A)
      ans = 0
    \end{verbatim}
\end{enumerate}
\subsection{Exercise Two}

\begin{enumerate}
  \item
  \begin{verbatim}
  > A = [1 13 40; 5 -3 8; -9 2 7];
  > B = [1 7 29; 4 6 10; 6 -3 0];
  > A*B
    ans =    
      293   -35   159
      41     -7   115
      41    -72  -241
  \end{verbatim}
  \item
      \begin{verbatim}
  > A = [1 0 0; 0 1 0; 0 0 1];
  > B = [1 2 3; 4 5 6; 7 8 9];
  > A*B
    ans =    
      1    2   3
      4    5   6
      7    8   9
  \end{verbatim}
\end{enumerate}
\subsection{Exercise Three}
 \begin{enumerate}
   \item
     \begin{verbatim}
  > A = [3 2 1; 0 7 2; 1 0 1];
  > B = [3 9 10];
  > A\B'
    ans =
      -1.5000
      -2.0000
      11.5000
     \end{verbatim}
  \item
     \begin{verbatim}
  > A = [0 1 2; 5 -2 2; 1 1 0];
  > B = [6 8 16];
  > A\B'
    ans =
       6.2500
       9.7500
      -1.8750
     \end{verbatim}
 \end{enumerate}
  
\end{document}